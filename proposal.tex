\documentclass[12pt]{amsart}
\usepackage[draft]{marktext} 
%% Remove draft for real article, put twocolumn for two columns
\usepackage[draft]{svmacro}
\usepackage[utf8]{inputenc}
\usepackage{lineno}
\usepackage[style=alphabetic, backend=biber]{biblatex}
\addbibresource{bibliography.bib}

%% commentary bubble
\newcommand{\SV}[2][]{\sidenote[colback=green!10]{\textbf{SV\xspace #1:} #2}}

%% Title 
\title{ Proposal for a class in asymptotic analysis }
%\author[1]{Co-author}
\author{Daniel Cooney}
\author{Daniel Gomez}
\author{Hyunjoon Kim}
\author{Truong-Son Van}
%\affil[1]{Institute}
\date{\today}

\begin{document}

\maketitle

We propose a class (series) in Asymptotic Analysis as part of the AMCS program at Penn,
aimed at advanced undergraduate and beginning graduate students.
Given the diverse background of AMCS students,
if the department is more ambitious about this class, it may be best splitting 
this course into two classes, the first deals with formal computations 
and important concepts/applications/motivations/simulations while the second
deals with completely rigorous justifications of computations introduced in the
first class.


\section{ Proposed syllabus}

\subsection*{References}

    We will write lecture notes based on the following texts
\begin{enumerate}
    \item Holmes, Introduction to perturbation methods~\cite{Holmes2013}
    \item Bensoussan, Lions and Papanicolaou, Asymptotic Analysis for Periodic Structures~\cite{BensoussanLionsPapanicolaou1978}
    \item Pavliotis and Stuart, Multiscale methods~\cite{PavliotisStuart2008}
    \item Hunter, Asymptotic Analysis and Singular Perturbation Theory~\cite{Hunter}
    \item Neu, Singular perturbation in the physical sciences~\cite{Neu2015}
    \item Keener, Principles of Applied Mathematics~\cite{Keener2000}
    \item Murray, Asymptotic analysis~\cite{Murray1984}
\end{enumerate}

\section*{Topics}

\subsection{First class}
For the first class, we propose the following topics:
\begin{enumerate}
    \item Asymptotic approximation 
        \begin{enumerate}
            \item Scaling, order relations, roots of equations
            \item Approximating integral: Laplace and steepest descent methods
        \end{enumerate}
    \item Regular perturbation theory
        \begin{enumerate}
            \item Power series expansion
        \end{enumerate}
    \item Singular perturbation theory
        \begin{enumerate}
            \item Boundary layers
            \item Vanishing coefficients 
        \end{enumerate}
    \item Averaging
        \begin{enumerate}
            \item Central limit theorems
            \item Ergodicity
        \end{enumerate}
    \item Homogenization theory
        \begin{enumerate}
            \item Effective diffusion, Cell problem
            \item Porous medium equation
        \end{enumerate}
\item Hydrodynamic limit 
    (this stuff could be skipped as it is 
    non-standard and will require a lot of thinking/planning)
        \begin{enumerate}
            \item Interacting particles, examples from statistical phsyics, mean-field games
            \item Propagation of chaos
        \end{enumerate}
\end{enumerate}

The instructor will have to come up with motivating examples from biology, physics,
economics, etc.

\subsection*{Second class}
For the second class, we propose the following topics:
\begin{enumerate}
    \item Stable and center manifolds
    \item Averaging and Ergodic theory
    \item KAM theory
    \item Homogenization theory
        \begin{enumerate}
            \item Two-scale convergence
            \item Rate of convergence
            \item Relation between ergodicity and periodic homogenization via scaling limit
        \end{enumerate}
    \item Weak KAM
\end{enumerate}
The second class is about rigorous justifications of some of the ideas from
the first class.



\printbibliography 
%\bibliography{refs}
%\bibliographystyle{halpha-abbrv}


\end{document}
